\makeglossaries

\newglossaryentry{amortisierung}{
    name=Amortisierung,
    description={Wann die Kosten wieder eingenommen werden\cite{amortisierung-lit}}
}

\newglossaryentry{anwendungsfall}{
    name=Anwendungsfall,
    description={Verhalten eines Systems unter bestimmten Bedingungen\cite{anwendungsfall-lit}}
}

\newglossaryentry{bare-metal}{
  name=Bare Metal,
  description={Physischer Server ohne Vorinstallierte Software\cite{bare-metal-lit}}
}

\newglossaryentry{branch}{
  name=Branch,
  description={Zweig in einer Softwareentwicklung\cite{branch-lit}}
}

\newglossaryentry{Centreon}{
  name=Centreon,
  description={Monitoring Tool für Netzwerke, Systeme und Anwendungen\cite{centreon-lit}}
}

\newglossaryentry{Checkmk}{
  name=Checkmk,
  description={Monitoring Anwendung zur Überwachung von IT-Infrastruktur\cite{checkmk-lit}}
}

\newglossaryentry{client}{
  name=Client,
  description={Computer, welches die Anwendung betreibt\cite{client-lit}}
}

\newglossaryentry{commit}{
  name=Commit,
  description={Einzelne Version eines Programms\cite{commit-lit}}
}

\newglossaryentry{cpu}{
  name=Central Processing Unit,
  description={Prozessor\cite{cpu-lit}}
}

\newglossaryentry{continuous-integration}{
  name=Continuous Integration,
  description={Softwareentwicklungsprozess wo neuer Code automatisch integriert wird\cite{continuous-integration-lit}}
}

\newglossaryentry{distribution}{
  name=Distribution,
  description={Betriebssysteme die auf Linux basieren\cite{distribution-lit}}
}

\newglossaryentry{feature}{
  name=Feature,
  description={Funktionalität eines Systems\cite{feature-lit}}
}

\newglossaryentry{framework}{
    name=Framework,
    description={Vorgefertigter Code als Grundgestell der Anwendung\cite{framework-lit}}
}

\newglossaryentry{gitlab}{
  name=GitLab,
  description={Open-Source Platform zur Softwareentwicklung\cite{gitlab-lit}}
}

\newglossaryentry{hochverfuegbarkeit}{
  name=Hochverfügbarkeit,
  description={Fähigkeit eines Systems über einen langen Zeitraum ohne Ausfälle zu laufen\cite{hochverfuegbarkeit-lit}}
}

\newglossaryentry{host}{
  name=Host,
  description={Computer, auf welchem die Checks laufen\cite{host-lit}}
}

\newglossaryentry{lint}{
  name=Linter,
  description={Software zur Code-Analyse\cite{lint-lit}}
}

\newglossaryentry{merge-request}{
  name=Merge-Request,
  description={in der Softwareentwicklung ein Vorschlag zu einem Code-Update\cite{merge-request-lit}}
}

\newglossaryentry{MetricQ}{
  name=MetricQ,
  description={hochskalierbares Framework zur Datenverarbeitung\cite{metricq-lit}}
}

\newglossaryentry{node}{
  name=Node,
  description={Verarbeitungseinheit in einem Rechnerverbund\cite{node-lit}}
}

\newglossaryentry{pipeline}{
  name=Pipeline,
  description={Automatisierte Prozesse der CI/CD\cite{pipeline-lit}}
}

\newglossaryentry{rack}{
  name=Rack,
  description={Gruppen von \acrshort{node}s\cite{rack-lit}}
}

\newglossaryentry{socket}{
  name=Socket,
  description={Steckplatz auf einer Hauptplatine für eine \acrshort{cpu}\cite{socket-lit}}
}

\newglossaryentry{tag}{
  name=Tag,
  description={Bezeichnung eines Objekts\cite{tag-lit}}
}

\newglossaryentry{unit-test}{
  name=Unit-Test,
  description={kleine Tests, welche eine einzelne Funktion oder Klasse testen\cite{unit-test-lit}}
}

\newglossaryentry{virtuelle-maschine}{
  name=Virtuelle Maschine,
  description={Emulation einer physischen Maschine\cite{virtuelle-maschine-lit}}
}

\newglossaryentry{ZIH-glos}{
  name=ZIH,
  description={Zentrum für Informationsdienste und Hochleistungsrechnen der Technischen Universität Dresden\cite{zih-lit}}
}

\newacronym{CI}{CI}{Continuous Integration}
\newacronym{CPU}{CPU}{Central Processing Unit}
\newacronym{EPK}{EPK}{Ereignisgesteuerte Prozesskette}
\newacronym{JSON}{JSON}{JavaScript Object Notation}
\newacronym{PDU}{PDU}{Power Distribution Unit}
\newacronym{TU}{TU}{Technische Universität}
\newacronym{VM}{VM}{Virtuelle Maschine}
\newacronym{ZIH}{ZIH}{Zentrum für Informationsdienste und Hochleistungsrechnen}
