\section{Einleitung}

\subsection{Projektumfeld}
An der Technischen Universität (\acrshort{TU}) Dresden bin ich am \acrlong{ZIH} (\gls{ZIH-glos}) beschäftigt.
Das \acrshort{ZIH} betreut zentrale IT-Dienste für Mitarbeitende, Studierende sowie andere Angehörige der Universität.
Zudem ist es für den Betrieb des Rechenzentrums verantwortlich.
Zur Echtzeitverarbeitung von Betriebs- und Leistungsdaten nutzt das Rechenzentrum der \acrshort{TU} Dresden das hochskalierbare \gls{framework} \gls{MetricQ}.
Über \gls{MetricQ} werden Metriken wie die Temperatur von Serverknoten, der Stromverbrauch von \acrshort{PDU}s oder sicherheitsrelevante Zustände (z.\,B. Türstatus) erfasst.
Diese Daten werden an ein zentrales Monitoring-System weitergeleitet, um einen umfassenden Überblick über den Systemzustand zu erhalten.
Aktuell wird hierfür \gls{Centreon} eingesetzt.
Da \gls{Centreon} \gls{MetricQ} nicht nativ unterstützt, ist eine aufwändige manuelle Konfiguration der Überwachungschecks erforderlich.

\subsection{Projektziel}
Ziel des Projekts ist die Entwicklung einer automatisierten Schnittstelle zwischen \gls{MetricQ} und dem neuen Monitoring-System \gls{Checkmk}.
Die Anwendung soll anhand von Metadaten und Namensschemata automatisch die IT-Infrastruktur abbilden und entsprechende Überwachungschecks generieren.
Änderungen an der Infrastruktur sollen automatisch erkannt werden, wodurch der manuelle Konfigurationsaufwand reduziert, Fehlerquellen minimiert und die Betriebssicherheit erhöht werden.

\subsection{Projektbegründung}
Die Durchführung des Projekts ist notwendig, da das bisher eingesetzte System \gls{Centreon} \gls{MetricQ} nicht nativ unterstützt.
Ein weiterer Grund ist, dass \gls{Centreon} durch \gls{Checkmk} abgelöst wird.
Die derzeitige manuelle Konfiguration der Überwachungschecks ist zeitaufwändig und fehleranfällig.
Durch die zu entwickelnde Schnittstelle werden diese Prozesse automatisiert, wodurch Fehler reduziert und die Betriebssicherheit signifikant erhöht wird.
Zusätzlich verbessert die Automatisierung die Effizienz im Rechenzentrumsbetrieb und spart langfristig Kosten und Zeit.

\subsection{Projektschnittstellen}
Die zu entwickelnde Dienstanwendung bildet die Schnittstelle zwischen \gls{MetricQ} und \gls{Checkmk}.
Sie kommuniziert mit \gls{MetricQ}, um auf Basis bereitgestellter Metadaten und Namensschemata die IT-Infrastruktur zu erfassen und generiert daraus automatisch Überwachungschecks für \gls{Checkmk}.
Neben dieser primären technischen Schnittstelle ist vorgesehen, dass die Anwendung von IT-System-Administratoren genutzt wird.
Diese sollen die Schnittstelle nutzten, um den Zustand des Rechenzentrums effizient zu überwachen und bei Bedarf Benachrichtigungen über notwendige Maßnahmen zu erhalten.

\subsection{Projektabgrenzung}
Das Projekt umfasst ausschließlich die Entwicklung der Schnittstelle zwischen \gls{MetricQ} und \gls{Checkmk}.
Nicht Bestandteil der Arbeit sind die vollständige Einführung oder Konfiguration von \gls{Checkmk} sowie Änderungen an der \gls{MetricQ}-Plattform.
Ebenso ist die Integration weiterer Systeme, die über die beiden genannten Anwendungen hinausgehen, nicht Teil des Projektumfangs.
