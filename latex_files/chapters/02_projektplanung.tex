\section{Projektplanung}

\subsection{Projektphasen}
Für die Umsetzung dieses Projekts inklusive Planung und Dokumentation stehen mir insgesamt 80 Stunden zur Verfügung.
Diese Zeit gliedere ich vor Beginn des Projektstarts in sechs Projektphasen.
Ich werde diese im Verlauf der Projektdurchführung wie folgt durchlaufen:

\begin{enumerate}
  \item Analyse
  \item Planung
  \item Realisierung
  \item Abnahme
  \item Projektdokumentation
  \item Kundendokumentation
\end{enumerate}

\noindent
Zusätzlich zu den sechs Phasen plane ich einen Zeitpuffer von zwei Stunden ein, um unvorhergesehene Ereignisse abzufangen.
Die zeitliche Verteilung aller Projektphasen ist in \reference{tab:zeitplanung} dargestellt.
Eine detailliertere Planung ist im Anhang unter \reference{tab:genaue-zeitplanung} zu finden.

\begin{table}[H]
  \centering
  \begin{tabular}{|l | c|}
    \hline
    \textbf{Projektphase} & \textbf{geplante Zeit} \\
    \hline
    Analyse                 & \qty{5}{\hour}  \\
    Planung                 & \qty{20}{\hour} \\
    Realisierung            & \qty{40}{\hour} \\
    Abnahme                 & \qty{2}{\hour}  \\
    Projektdokumentation    & \qty{7}{\hour}  \\
    Kundendokumentation     & \qty{4}{\hour}  \\
    Pufferzeit              & \qty{2}{\hour}  \\
    \hline
    \textbf{Gesamt}         & \qty{80}{\hour} \\
    \hline
  \end{tabular}
  \caption{Zeitplanung}
  \label{tab:zeitplanung}
\end{table}

\subsection{Ressourcenplanung}
Für die erfolgreiche Umsetzung des Projekts benötige ich verschiedene Ressourcen.
Diese setzen sich aus Hard- und Softwarekomponenten sowie aus personellen Ressourcen zusammen.
Zur effizienten Gestaltung der Projektkosten wird darauf geachtet, bestehende Softwarelösungen bestmöglich wiederzuverwenden.
Eine vollständige Auflistung aller verwendeten Ressourcen befindet sich im Anhang in \reference{tab:verwendete-ressourcen}.

\subsection{Entwicklungsprozess}
Für die Entwicklung verwende ich eine agile Vorgehensweise.
Dabei stehe ich in regelmäßigen Abständen in Kontakt mit dem Auftraggeber und dem Kunden, um sicherzustellen, dass das Projekt den Anforderungen entspricht.
Diese kontinuierlichen Abstimmungen ermöglichen es, flexibel auf neue Anforderungen zu reagieren und Lösungen schrittweise zu entwickeln.
Während der Realisierungsphase programmiere ich iterativ und verbessere Funktionen fortlaufend basierend auf dem erhaltenen Feedback.
