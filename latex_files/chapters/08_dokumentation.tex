\section{Dokumentation}
Die Dokumentation des Projekts umfasst mehrere Bestandteile, die dem Kunden und den Entwicklern helfen, die Anwendung zu verstehen und zu nutzen.
Die \textbf{Codedokumentation} wurde direkt im Code implementiert, indem alle Methoden und Funktionen mithilfe von \textit{Docstrings} ausführlich dokumentiert wurden.
Diese Docstrings erläutern die Funktionsweise der einzelnen Bestandteile des Codes und ermöglichen eine einfache Wartung sowie Erweiterung der Anwendung.
Zusätzlich befindet sich im \textit{GitLab-Repository} eine \textbf{einfache Kundendokumentation}, die als \textit{README}-Datei im Markdown-Format vorliegt.
Diese Dokumentation bietet eine schnelle Einführung in die grundlegenden Funktionen der Anwendung und erklärt dem Kunden die wichtigsten Anwendungsmöglichkeiten.
Für eine tiefere Auseinandersetzung mit der Anwendung steht eine \textbf{ausführliche Kundendokumentation} zur Verfügung.
Diese befindet sich im Abschnitt \textit{B Kundendokumentation} und bietet eine detaillierte Beschreibung der Anwendung, ihrer Funktionen sowie der Konfiguration und Nutzung.
