\section{Fazit}

\subsection{Soll-/Ist-Vergleich}
Das Projektziel wurde größtenteils erreicht, abgesehen von den Anpassungen, die in \reference{sec:implementierungsphase-probleme} beschrieben wurden.
Die Anforderungen des Kunden, wie das Anzeigen der Checks in \Gls{Checkmk} und deren Anbindung an \Gls{MetricQ}, wurden vollständig erfüllt.
Der Auftraggeber ist zufrieden, und die geplante Arbeit wurde weitgehend eingehalten, bis auf kleinere Abweichungen.
Eine Tabelle zum Vergleich der tatsächlichen und geplanten Zeiten findet sich im Anhang unter \reference{tab:zeitvergleich}.
Das Budget wurde nicht vollständig ausgeschöpft, da weniger Teammeetings und Fachliche Abstimmungen stattfanden, als geplant.
Eine entsprechende Kostenaufstellung zeigt eine Einsparung von \qty{166}{\euro}.

\begin{table}[H]
  \centering
  \small  % kleinere Schrift
  \renewcommand{\arraystretch}{1.1}  % geringerer Zeilenabstand
  \begin{tabular}{
      >{\raggedright\arraybackslash}p{5.2cm}
      >{\centering\arraybackslash}p{2.2cm}
      >{\raggedright\arraybackslash}p{5.5cm}
      >{\centering\arraybackslash}p{2.4cm}
    }
    \toprule
    \textbf{Tätigkeit} & \textbf{Dauer [h]} & \textbf{Stundensätze} & \textbf{Gesamt [€]} \\
    \midrule
    Projektbearbeitung
    & 80
    & \SI{7}{\euro} (Pers.) + \SI{20}{\euro} (Ress.)
    & \num{2160} \\

    Teammeetings
    & 5
    & 2 × \SI{27}{\euro} + \SI{20}{\euro}
    & \num{370} \\

    Fachliche Abstimmung
    & 4
    & \SI{7}{\euro} + 3 × \SI{27}{\euro} + \SI{20}{\euro}
    & \num{432} \\
    \midrule
    \multicolumn{3}{r}{\textbf{Gesamtkosten:}} & \textbf{\num{2962}} \\
    \bottomrule
  \end{tabular}
  \caption{Kostenaufstellung für das Projekt}
  \label{tab:final-kostenrechnung}
\end{table}

\subsection{Lessons Learned}
Während des Projekts habe ich wertvolle Lektionen gelernt.
Die Anforderungen des Auftraggebers wurden klarer, was die Umsetzung erleichterte.
Eine gute Planung war entscheidend für den Projekterfolg, da unvorhergesehene Probleme leicht zu Verzögerungen führen können.
Ich habe auch gelernt, wie unterschiedlich die Dokumentationsqualität der verwendeten Technologien sein kann und wie wichtig eine realistische Zeitplanung ist.

\subsection{Ausblick}
In Zukunft könnte das Projekt weiter verbessert werden.
Eine neue Planungsphase oder eine separate Anwendung zur Automatisierung der Struktur könnte die verbleibenden manuellen Konfigurationen weiter reduzieren und die Fehleranfälligkeit verringern.
Außerdem könnte die Anwendung um weitere Check-Arten erweitert werden, um eine größere Bandbreite an Fehlern zu erkennen und die Fehlererkennung zu verbessern.
